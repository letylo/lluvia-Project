\section{Conclusiones, trabajos futuros y posibles aplicaciones}
\label{section:conclusiones}
\noindent\textbf{Conclusiones}

La intención a la hora de realizar un trabajo como este, era la de recrear un sistema  capaz de simular comportamientos emergentes capaces de imitar la conducta de manada de ciertos animales según diferentes contextos. 

Las conclusiones a las que se han llegado con respecto a este tema son : 
\begin{enumerate}
 \item Al tratarse de comportamientos emergentes, aunque se repitan los parámetros de la simulación rara vez se repiten situaciones.
 \item Los tres compartimientos principales de manada(alineación, separación y cohesión) deben estar en perfecto equilibrio entre ellos. Cualquier pequeño desajuste echa por tierra todo el modelo computacional. 
 \item Modificando solo algunos parámetros de los nanobots se pueden simular distintos tipos de manada: bancos de peces, bandadas de aves...
 \item El campo de visión es el parámetro mas influyente a la hora de estudiar los comportamientos, si se reduce el ángulo se tienden a agrupar en hileras, mientras que, si se amplia se agrupan en bancos.   
 \item Se añadió, al final del desarrollo, la variable velocidad de crucero mejorando considerablemente el funcionamiento de las manadas.
 \item Dados los malos resultados obtenidos con la lista de dinámica para la selección de nanobots y speakers se desaconseja su utilización para otros proyectos de carácter similar.
\end{enumerate}

En cuanto a la parte de audio se refiere, ha tenido un trabajo de estudio y documentación
exigente. Era necesario poder hacerlo pensando en programación web, e intentar aplicar las
últimas tecnologías que proporciona: HTML5, CSS3 y Javascript.
Web Audio API, es una biblioteca JS relativamente novedosa, y específicamente en Mozilla
Firefox se han incorporado nuevas funciones que antes no eran compatibles con este
navegador
https://blog.mozilla.org/blog/2014/04/29/mozilla-introduces-the-most-
customizable-firefox-ever-with-an-elegant-new-design/
\\

Web Audio API es una biblioteca muy potente que cuenta con infinidad de posibilidades. Esto
implica, que no hay una sola manera de poder hacer las cosas. La documentación se encuentra
en su mayoría en inglés, pero es muy orientativo, y acerca al poder que tiene esta API, y 
como empezar a tomar ideas.\\

Como conclusión final se puede decir que la vida(en nuestro caso artificial) es terriblemente susceptible a pequeños cambios dentro de su propio equilibrio y que la mejor forma de preservar ese equilibrio es atenerse a unas simples reglas de comportamiento.

Por esta razón es importante tener el mayor conocimiento posible, acerca de las cualidades y comportamientos de los seres vivos antes de emprender un proyecto de estas característica.\\


\noindent\textbf{Trabajos futuros}

En futuras versiones, se
 plantearán las siguientes cuestiones:
\begin{itemize}
 \item Implementación de pequeños efectos de audio en tiempo real o pequeñas demos acústicas.
 \item Profunda optimización del código fuente para poder mostrar mas nanobots por pantalla.
 \item Dotar a los nanobots del sentido del olfato.
 \item Implementar el concepto de depredador y de dirección del viento para ampliar el campo de investigación para comportamientos de caza entre animales. 
\end{itemize}



\noindent\textbf{Posibles aplicaciones}

Las aplicaciones mas inmediatas son las simulaciones de todo tipo que tengan como base obtener el comportamiento global de un gran número de agentes autónomos que interactúan entre si. Por ejemplo, simulaciones de trafico o manifestaciones, conjuntos de células, comunicación entre neuronas...

Ampliando un poco el área de aplicación podría adaptarse con relativa facilidad para poder ser utilizado en películas y videojuegos. 
